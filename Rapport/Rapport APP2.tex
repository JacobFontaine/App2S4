\documentclass[12pt]{article}
\usepackage[margin=3.0cm]{geometry}
\usepackage[french, english]{babel}
\usepackage[utf8]{inputenc}
%\usepackage[hidelinks]{hyperref}
\usepackage{graphicx}
\usepackage{amsmath}
\usepackage{graphicx}
\graphicspath{{Images/}
\geometry{legalpaper}}

\begin{document}
\begin{titlepage} 
	\large
	{
		\begin{center}
			UNIVERSITÉ DE SHERBROOKE\\Faculté de génie\\
			Département de génie électrique et génie informatique\\
			\vspace{3cm}
			{\LARGE\textbf{Principes de dynamique et méthodes numériques}}\\
			\vspace{2cm}
			\LARGE{Rapport APP2}\\
			\vspace{2cm}
			Présenté à\\l'équipe professorale de la session S4\\
			\vspace{2cm}
			Produit par\\Axel Bosco, Jacob Fontaine, Philippe Spino\\
			\vspace{1cm}
			\vfill{23 mai 2017 - Sherbrooke}
		\end{center}
	}
\end{titlepage}
\tableofcontents
\newpage
\section{Introduction}
Principes de dynamique et méthodes numériques
Dans le cadre du cour \textit{Principes de dynamique et méthodes numériques}, le mandat remit à la présente équipe était de rendre l'initiation des étudiants de la faculté de génie plus passionnante a l'aide d'un parcours à obstacles de style \textit{Wipe-out}.

\section{Design de la glissade}
Le devis émit par le WOQ demandais de calculer la trajectoire d'un glissade passant absolument par des points cartésiens précis. Comme le montre la tableau $1$, la glissade doit passer par tout les points.

\begin{center}
	\includegraphics[width=\textwidth]{tableau1}
\end{center}
Les valeurs de ce tableau sont les points de référence pour le approximation de la courbe de la glissade. Il faut approximer une courbe qui passe par tous les points. Ensuite, il faut trouver la bonne courbe en évaluant toutes les courbes possibles des points $E_y$. Les valeurs possibles du points $E_y$ vont de 10 à 15. Les coordonnées horizontales$(m)$ sont exprimés par la suite $X_k$ et les coordonnées verticales$(m)$ sont exprimées par la suite $Y_k$. 
\subsection{Calculs}
Il y a 5 termes par suites mentionnés précédemment. Le devis demander d'interpoler les valeurs des coordonnées. L'interpolation veut dire que notre nombre de termes dans les suites, soit M, est égale au nombre de termes du polynôme, soit $N=M$. En utilisant l'appoximation linéaire, on obtien un polynome d'ordre 4 (l'ordre est $N-1$). Le polynôme obtenu suite a l'interpolation est: 
\begin{equation}
g(x) = A(1) + A(2)\times x + A(3)\times x^2 + A(4)\times x^3 + A(5)\times x^4 
\end{equation}
Où $A$ est le vecteur des coefficient des polynômes obtenu par la méthode de la projection orthogonale avec la pseudo-inverse. 
\newpage
\noindent
Une fois les polynômes trouver ($E_y$ allant de 10 à 15 par bon de 0,01), il faut choisir une des courbe qui assure la sécurité des participants. Le devis mentionne que: "La valeur de yf doit assurer une sortie de glissade à peu près horizontale (exigence qualitative) pour assurer une transition naturelle avec la surface horizontale de la trappe." On peut alors dire qu'il faut que la dérivé du polynôme soit le plus près de 0. 
\begin{equation}
\frac{dg(x)}{dx} = 0
\end{equation}
L'estimation a été accomplie avec le logiciel de calcul Matlab et le résultat obtenu est que la valeur de $E_y = 12.27m$. Ce qui donne la courbe présenté sur la figure 1.

\begin{center}
	\includegraphics[width=\textwidth]{figure1}
\end{center}


\section{Design du débit d'eau}
Pour trouver le débit d'eau du parcours, il faut trouver la vitesse du participant sans frottement pour déterminer s'il faut y appliquer un coefficient de frottement et si c'est nécessaire, il faut déterminer ce coefficient.
\subsection{Calculs}
\subsubsection{Vitesse sans frottement}
Afin de valider si le participant sortent dans la bonne tranche de vitesse, soit entre $20km/h$ et $25km/h$, il faut d'abord vérifier sa vitesse de sortie sans frottement. Le théorème d'énergie mécanique sans frottement correspond à celui-ci, 
\begin{equation}
T_i + V_{gi} = T_f + V_{gf}
\end{equation}
Sachant que,
\begin{equation}
g = 9.81m/s^2
\end{equation}
\begin{equation}
m = 80kg
\end{equation}
\begin{equation}
h_0 = 30m
\end{equation}
\begin{equation}
h_f = 12,27m
\end{equation}
\begin{equation}
v_i = 0m/s
\end{equation}
\begin{equation}
v_f = ?m/s
\end{equation}
Et que,
\begin{equation}
T_i = \frac{1}{2} \times m \times v_i^2
\end{equation}
\begin{equation}
T_f = \frac{1}{2} \times m \times v_f^2
\end{equation}
\begin{equation}
V_{gi} = m \times g \times h_0
\end{equation}
\begin{equation}
V_{gf} = m \times g \times h_f
\end{equation}
Au début du parcours, la vitesse du participant est nulle, $v_i = 0$, donc $T_i = 0$. Le théorème devient,
\begin{equation}
V_{gi} = T_f + V_{gf}
\end{equation}
En substituant les équations des énergies dans le théorème, l'équation devient,
\begin{equation}
m \times g \times h_0 = \frac{1}{2} \times m \times v_f^2 + m \times g \times h_f
\end{equation}
En isolant $v_f$, l'équation devient,
\begin{equation}
v_f = \sqrt{\frac{2 \times (m \times g \times h_0 - m \times g \times h_f)}{m}}
\end{equation}
En évaluant l'équation, la vitesse du participant est donnée en $m/s$, donc il faut multiplier la valeur par $3,6$ afin d'avoir des $km/h$. La vitesse au point E est la suivante,
\newline
\begin{center}
	\includegraphics[scale=0.8]{valeurvfsansfrottement}
\end{center}
Le participant va à $v_f = 67,14km/h$ en arrivant au point E, beaucoup plus rapide que demandé. Afin d'y remédier, il faut trouver un coefficient de frottement.
\subsubsection{Ordre et coefficients du polynôme d'approximation de $\mu_{f}$}
Différents coefficients de frottement sont obtenus selon le pourcentage d'ouverture de la valve d'eau. Afin de trouver le pourcentage d'ouverture de la valve pour avoir le coefficient de frottement $\mu_f$ obtenu dans la prochaine section, il faut trouver l'ordre et les coefficients du polynome d'approximation.
\newline
\newline
Les valeurs connues de la relation entre l'ouverture de la valve et le coefficient sont indiquées dans le tableau 2 suivant
\begin{center}
	\includegraphics[scale=0.8]{tableau2}
\end{center}
Puisqu'il s'agit d'une approximation, la relation $N - M \geq 3$ doit être vraie en tout temps. Les polynômes d'approximation peuvent aller de l'ordre $0$ à l'ordre $M - 1$. Puisque,
\begin{equation}
N - M \geq 3
\end{equation}
\begin{equation}
M_{max} = N - 3
\end{equation}
\begin{equation}
M_{max} = 8
\end{equation}
Donc, l'ordre maximal est de 8.
\newline
\newline
Suite à ceci, puisqu'il y a plusieurs possibilités d'ordre, il faut trouver quel polynôme est le plus précis possible. Le $RMS$ et le facteur $R$ sont des paramètres permettant de déterminer la validité d'une approximation. Le $RMS$ est la marge d'erreur entre des valeurs exactes et des valeurs approximées au même endroit. Plus le $RMS$ est petit, moins l'erreur est grande. Le facteur $R$ est un facteur qui indique l'exactitude du polynôme et s'il passe le plus possible à travers tous les points précis. Plus sa valeur est proche de $1$, plus l'approximation est exacte.
\newline
\begin{center}
	\includegraphics[width=\textwidth]{polynome}
\end{center}
Selon les paramètres $RMS$ et $R$, les polynômes $M = 4$ et $M = 5$ sont les plus précis. Le polynôme $M = 5$ est celui sélectionné pour le reste de la problématique, puisqu'il a une plus petite erreur sur les valeurs pour une perte de précision minime sur le polynôme d'approximation.
\newline
\newline
Donc, le polynôme sélectionné est celui à $M = 5$ d'ordre $4$. Le polynôme est représenté par l'équation suivante,
\begin{equation}
g(x) = A(1) + A(2)\times x + A(3)\times x^2 + A(4)\times x^3 + A(5)\times x^4 
\end{equation}
\subsubsection{Coefficient de frottement $\mu_{f}$}
Pour trouver le coefficient, il faut repenser au théorème d'énergie mécanique. Par définition, le frottement est une force qui s'oppose au mouvement entre deux systèmes. Dans le cas du parcours, les deux systèmes sont le participant et la glissade. Pour illustrer le principe du frottement, voici un DCL,
\begin{center}
	\includegraphics[scale=0.8]{dcl}
\end{center}
Le carré représente le participant et la surface plane en dessous représente la glissade. Comme la définition le dit, un frottement est une force qui s'oppose au déplacement. Il existe deux types de frottements : statique et cinétique. Le frottement statique est la force qui s'oppose complètement au déplacement, donc l'objet ne possède aucune vitesse. Le frottement cinétique est celui qui s'oppose au mouvement ayant une vitesse. Puisque le participant glisse, la force de frottement est opposé à sa vitesse.
\newline
\newline
Une force de frottement cinétique peut être décrite par l'équation suivante,
\begin{equation}
f_c = \mu_fN
\end{equation}
Où, $\mu_f$ représente son coefficient de frottement cinétique et $N$ représente la normale de l'objet. Donc, sachant que $N$ représente la normale de l'objet, l'équation de la force devient,
\begin{equation}
f_c = \mu_fmg
\end{equation}
Puisque,
\begin{equation}
N = mg
\end{equation}
Pour pouvoir adapter le théorème d'énergie mécanique avec la force de frottement, il faut convertir cette force en énergie. Cette énergie est le travail. Or, le travail est par définition, une force sur un déplacement.
\begin{equation}
U_{1,2}^{'} = \vec{F} \cdot \vec{d}
\end{equation}
Avec cette nouvelle relation, le théorème devient,
\begin{equation}
T_i + V_{gi} + U_{1,2}^{'} = T_f + V_{gf}
\end{equation}
En isolant $U_{1,2}^{'}$ et puisque $v_i = 0$, donc $T_i = 0$, le théorème devient,
\begin{equation}
U_{1,2}^{'} = T_f + V_{gf} - V_{gi}
\end{equation}
En substituant les équations dans le théorème, l'équation devient,
\begin{equation}
\mu_fmgx = \frac{1}{2}mv_f^2 + mgh_f - mgh_i
\end{equation}
Puisque le devis demande le coefficient pour une vitesse de sortie de $22,5km/h$, donc $6,25m/s$, $\mu_c$ peut être isolé,
\begin{equation}
\mu_f = \frac{\frac{1}{2}mv_f^2 + mgh_f - mgh_i}{mgx}
\end{equation}
L'équation simplifiée devient,
\begin{equation}
\mu_f = \frac{\frac{1}{2}v_f^2 + gh_f - gh_i}{gx}
\end{equation}
En évaluant avec les valeurs connues,
\begin{center}
	\includegraphics[scale=0.8]{valeurmuf}
\end{center}
\subsubsection{Pourcentage d'ouverture de la valve}
Puisque le coefficient de frottement $\mu_f$ est connu et son polynôme d'approximation avec ses coefficients d'approximation, il est possible de déterminer la valeur du pourcentage d'ouverture de la valve.
\newline
\newline
Pour se faire, il faut résoudre l'équation suivante,
\begin{equation}
g(x) = A(1) + A(2)\times x + A(3)\times x^2 + A(4)\times x^3 + A(5)\times x^4
\end{equation}
Avec,
\begin{equation}
g(x) = 0,6296
\end{equation}
Donc,
\begin{equation}
0,6296 = A(1) + A(2)\times x + A(3)\times x^2 + A(4)\times x^3 + A(5)\times x^4
\end{equation}
À l'aide de Matlab, l'équation résolue donne les valeurs suivantes,
\begin{center}
	\includegraphics[scale=0.8]{ouverture}
\end{center}
Puisque la valve ne peut pas avoir une ouverture supérieure à $100\%$, la valeur de l'ouverture pour un coefficient $\mu_f = 0,6296$ est de $30,9459\%$.
\subsubsection{Vitesse avec frottement du participant durant le parcours}
\begin{center}
	\includegraphics[width=\textwidth]{figure2}
\end{center}


\section{Design du Ballon-mousse}
\subsection{Ballon Attrapé}
Dans cette situation, on présume que le participant attrape le ballon-mousse. Donc, on peut assumer alors qu'il y a une fusion du ballon-mousse et le participant après l'impacte en ceux-ci? Donc cela se résume à l'équation suivante:
\begin{equation}
m_p\times v_p + m_b\times v_b = (m_p + m_b)\times v_{pb}
\end{equation}
En isolant $v_{pb}$, on obtien un valeur de :
\begin{equation}
v_{pb} = 5,59m/s
\end{equation}
À l'aide de cette vitesse, on doit règler la minutrie en sorte à ce que le participant ait quitté la plateform au complet avant que celle-ci s'ouvre.
\begin{equation}
\delta t_m = \frac{l_{trappe}}{v_{pb}}
\end{equation}
\begin{equation}
\delta t_m = \frac{3m}{5,59m/s} \approx{0,54}
\end{equation}
Et selon les standard imposés, la minutrie devait avoir une marge de manoeuvre de $0,02s$.
\begin{equation}
\delta t_m \approx{0,54 - 0,02} = 0,52sec
\end{equation}
\subsection{Ballon non attrapé}
Dans cette situation, le participant entre en collision avec le ballon sans l'attrapé. La collision entre le ballon-mousse et le participant à ce moment là est une collision plastique. Selon les requis du devis de WOQ, nous considérons le coefficient de récupération de $0,8$. Les valeurs de $V_{p_n}$ = 6.25$m/s$ et $V_{b_n}$ = -1.0$m/s$
\begin{equation}
e <= 0,8 = \frac{V'_{b_n} - V'_{p_n}}{V_{p_n} - V_{b_n}}
\end{equation}
suite a des manipulations algébrique, le résultat est:
\begin{equation}
{V_{b_n} - V_{p_n}} = 5,8
\end{equation}
\begin{equation}
V_b' = 5,8 + V_p'
\end{equation}
Il y a aussi l'équation suivante:
\begin{equation}
m_p\times V_p + m_b\times V_b = m_p\times V_p' + m_b\times V_b'
\end{equation}
\noindent
en substituant l'équation trouvé précédement, on obtien:
\begin{equation}
V_p' = \frac{m_p.V_p + m_b.V_b}{(m_b + m_p).V_b'}
\end{equation}
\begin{equation}
V_p' = 5.06m/s
\end{equation}
Donc, le temps requis pour traverser complètement la trappe est:
\begin{equation}
t_m = \frac{l_{trappe}}{V_p'} = 0,59 sec
\end{equation}

\section{Design de la minuterie}
Le devis avait un requis d'une marge de maneuvre de $0,02sec$ pour assurer la sécurité des participants concernant la trappe. Lorsque le ballon est attrapé, les temps est de $0,52sec$ et lorsque le ballon rebondit, le temps est de $0,59sec$. $\Delta T_m$ est le temps pour peu importe le cas avec le participant et le ballon.
\subsection{Calculs}
\begin{equation}
t_{m_f} < \Delta T_m - 0,02
\end{equation}
\begin{center}
et
\end{center}
\begin{equation}
t_{m_r} > \Delta T_m + 0,02
\end{equation}

Suite aux manipulations algébriques, le résultat est que 
\begin{equation}
\Delta T_m \approx 0,056 sec
\end{equation}

\section{Design du Coussin-Tampoline}
En ce qui concerne le design du Coussin-Trampoline, le devis fournit toutes les informations nécessaires pour déterminer la déformation totale du coussin-trampoline. Sachant que $k_c = 6000N/m$ et que le participant-ballon, d'une masse de $m = 88kg$ tombe d'une hauteur $h_0$ de $5 m$. Sachant que le coussin va avoir une déformation de $h_c$, la chute totale du participant-ballon est de $\Delta h = h_0 + h_c$. La vitesse horizontale du participant-ballon n'affecte pas sa vitesse verticale, seule la gravité va affecter sa vitesse.
\subsection{Calculs}
Sachant que,
\begin{equation}
T_i + V_{gi} + V_{ri} = T_f + V_{gf} + V_{rf}
\end{equation}
Où,
\begin{equation}
T_i = 0
\end{equation}
\begin{equation}
V_{ri} = 0
\end{equation}
\begin{equation}
T_f = 0
\end{equation}
\newline
Au début de la chute, la vitesse verticale du participant-ballon est nulle. À la fin de la chute, la vitesse verticale du participant est nulle, puisque le coussin-trampoline a amorti la chute. Aussi, la déformation du coussin-trampoline est nulle, donc il n'y a aucune énergie potentielle de ressort. 
\newline
\newline
Donc, le théorème d'énergie mécanique après simplification devient,
\begin{equation}
V_{gi} = V_{gf} + V_{rf} 
\end{equation}
Sachant que,
\begin{equation}
V_{gi} = m \times g \times h_0 
\end{equation}
\begin{equation}
V_{gf} = m \times g \times h_c 
\end{equation}
\begin{equation}
V_{rf} = \frac{1}{2} \times k_c \times h_c^2
\end{equation}
\newline
En substituant les équations de chacun des composants du théorème, l'équation devient,
\begin{equation}
m \times g \times h_0 = m \times g \times h_c + \frac{1}{2} \times k_c \times h_c^2
\end{equation}
\newline
\newline
Il est possible de transformer cette équation en équation quadratique.
\begin{equation}
0 = \frac{1}{2} \times k_c \times h_c^2 + m \times g \times h_c - m \times g \times h_0
\end{equation}
\newline
En résolvant l'équation quadratique, la valeur de $h_c$ est obtenue.
\subsection{Résultats}
La résolution d'une quadratique donne deux valeurs. Les deux valeurs obtenues sont affichées ci-bas.
\begin{center}
	\includegraphics[scale=0.8]{valeurhc}
\end{center}
Puisqu'il s'agit d'une compression d'un ressort, la valeur de la distance de compression doit être négative. Donc, la valeur de $h_c = -1,3520m$.
\section{Design du Bassin}
\subsection{Calculs}
L'équation différentielle linéarisée sera faite avec la méthode longue, car l'équation d'équilibre est demandé. Les étapes seront donc détaillés ci-dessous.
\newline
\newline
Étape 1: Équation à l'équilibre: On pose tout d'abord notre équation à linéariser: 
\begin{equation}
m\frac{\delta v}{\delta t} = K_fmg-mg+BV^2
\end{equation}
On transforme notre dérivé selon z, et on simplifie:  
\begin{equation}
m\frac{\delta v}{\delta z} = \frac{mg}{V}(K_f-1)+BV
\end{equation}
Par la suite, on met l'équation en équilibre:
\begin{equation}
0 = \frac{mg}{V_E}(K_f-1)+BV_E
\end{equation}
Étape 2: Linéarisation des termes non linéaires avec une série de Taylor d'ordre 1. On a un seul terme non linéaires, et une seule sortie. Donc:
\begin{equation}
\frac{mg}{V}(K_f-1) = \frac{mg}{V_E}(K_f-1) - (\frac{mg}{V_E^2}(K_f-1)(V-V_E)
\end{equation}
Étape 3: On remplacement maintenant les termes linéarisés dans l'équation originale:
\begin{equation}
m\frac{\delta v}{\delta z} = \frac{mg}{V_E}(K_f-1) - (\frac{mg}{V_E^2}(K_f-1)(V-V_E)+BV
\end{equation}
Étape 4: On change alors les variables en terme de variation autours de l'équilbre. En sachant que :
\begin{equation}
V= V_E + \Delta V
\end{equation}
On pose: 
\begin{equation}
m\frac{\delta (V_E + \Delta V)}{\delta z} = \frac{mg}{V_E}(K_f-1) - (\frac{mg}{V_E^2}(K_f-1)((V_E + \Delta V)-V_E)+B(V_E + \Delta V)
\end{equation}
\begin{equation}
m\frac{\delta \Delta V}{\delta z} = \frac{mg}{V_E}(K_f-1) - (\frac{mg}{V_E^2}(K_f-1) \Delta V)+BV_E + B\Delta V
\end{equation}
Étape 5: Finalement, on soustrait l'équation d'équilibre:
\begin{equation}
m\frac{\delta \Delta V}{\delta z} = - (\frac{mg}{V_E^2}(K_f-1) \Delta V)+ B\Delta V
\end{equation}
Ceci est donc notre équation différentielle linéarisée.
\newline
\newline
Pour trouver $V_E$, on prend notre équation à l'équilibre, et on isole $V_E$.
\begin{equation}
0 = \frac{mg}{V_E}(K_f-1)+BV_E
\end{equation}
\begin{equation}
V_E = \sqrt{\frac{-mg(K_F-1)}{B}}
\end{equation} 
\begin{equation}
V_E = 0.9137m/sec
\end{equation}  
Finalement, on recherche la profondeur du bassin. Donc en partant de notre équation différentielle linéarisée, on isole les $\Delta V$ et les z: 
\begin{equation}
m\frac{\delta \Delta V}{\delta z} = - (\frac{mg}{V_E^2}(K_f-1) \Delta V)+ B\Delta V
\end{equation}
\begin{equation}
\frac{\delta \Delta V}{\Delta V} =  (-\frac{g}{V_E^2}(K_f-1)+ \frac{B}{m})\delta z
\end{equation}
Nous intégrons maintenant les 2 côtés:
\begin{equation}
\int_{0.1V_E}^{V_I-V_E}\frac{\delta \Delta V}{\Delta V} = \int_{0}^{z_f} (-\frac{g}{V_E^2}(K_f-1)+ \frac{B}{m})\delta z
\end{equation}
\begin{equation}
\int_{0.1V_E}^{V_I-V_E}\frac{\delta \Delta V}{\Delta V} = (-\frac{g}{V_E^2}(K_f-1)+ \frac{B}{m})z_f
\end{equation}
La variable $V_I$ représente la vitesse initiale du participant lorsqu'il entre dans l'eau. En sachant que la hauteur h est de 10m, il est calculé grâce à:
\begin{equation}
V_I = \sqrt{2gh}=14 m/s
\end{equation}
$\Delta z$ est alors calculé grâce à MatLab, et le résultat de la profondeur du bassin est de 4.23m
\section{Conclusion}


\end{document}
