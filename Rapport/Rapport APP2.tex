\documentclass[12pt]{article}
\usepackage[margin=3.0cm]{geometry}
\usepackage[french, english]{babel}
\usepackage[utf8]{inputenc}
\usepackage{amsmath}
\usepackage{setspace}

\begin{document}
\begin{titlepage}   
	\large
	{
		\begin{center}
			UNIVERSITÉ DE SHERBROOKE\\Faculté de génie\\
			Département de génie électrique et génie informatique\\
			\vspace{3cm}
			{\LARGE\textbf{Principes de dynamique et méthodes numériques}}\\
			\vspace{2cm}
			\LARGE{Rapport APP2}\\
			\vspace{2cm}
			Présenté à\\l'équipe professorale de la session S4\\
			\vspace{2cm}
			Produit par\\Axel Bosco, Jacob Fontaine, Philippe Spino\\
			\vspace{1cm}
			\vfill{23 mai 2017 - Sherbrooke}
		\end{center}
	}
\end{titlepage}

\tableofcontents
\newpage
\onehalfspacing
\section{introduction}
Principes de dynamique et méthodes numériques
Dans le cadre du cour \textit{Principes de dynamique et méthodes numériques}, le mandat remit à la présente équipe était de rendre l'initiation des étudiants de la faculté de génie plus passionnante a l'aide d'un parcours à obstacles de style \textit{Wipe-out}.

\section{Design de la glissade}
\subsection{Calcul}

\section{Design du débit d'eau}
\subsection{Calcul}

\section{Design du Ballon-mousse}
\subsection{Calcul}

\section{Design de la minuterie}
\subsection{Calcul}

\section{Design du Coussin-Tampoline}
\subsection{Calcul}

\section{Design du Bassin}
\subsection{Calcul}

\end{document}