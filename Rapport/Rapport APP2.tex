\documentclass[12pt]{article}
\usepackage[margin=3.0cm]{geometry}
\usepackage[french, english]{babel}
\usepackage[utf8]{inputenc}
%\usepackage{hyperref}
\usepackage{amsmath}
%\usepackage{setspace}

\begin{document}
\begin{titlepage} 
	\large
	{
		\begin{center}
			UNIVERSITÉ DE SHERBROOKE\\Faculté de génie\\
			Département de génie électrique et génie informatique\\
			\vspace{3cm}
			{\LARGE\textbf{Principes de dynamique et méthodes numériques}}\\
			\vspace{2cm}
			\LARGE{Rapport APP2}\\
			\vspace{2cm}
			Présenté à\\l'équipe professorale de la session S4\\
			\vspace{2cm}
			Produit par\\Axel Bosco, Jacob Fontaine, Philippe Spino\\
			\vspace{1cm}
			\vfill{23 mai 2017 - Sherbrooke}
		\end{center}
	}
\end{titlepage}
\tableofcontents
\newpage
%\onehalfspacing
\section{Introduction}
Principes de dynamique et méthodes numériques
Dans le cadre du cour \textit{Principes de dynamique et méthodes numériques}, le mandat remit à la présente équipe était de rendre l'initiation des étudiants de la faculté de génie plus passionnante a l'aide d'un parcours à obstacles de style \textit{Wipe-out}.

\section{Design de la glissade}
\subsection{Calculs}

\section{Design du débit d'eau}
\subsection{Calculs}

\section{Design du Ballon-mousse}
\subsection{Ballon Attrapé}
Dans cette situation, on présume que le participant attrape le ballon-mousse. Donc, on peut assumer alors qu'il y a une fusion du ballon-mousse et le participant après l'impacte en ceux-ci? Donc cela se résume à l'équation suivante:
\begin{equation}
m_p*v_p + m_b*v_b = (m_p + m_b)*v_{pb}
\end{equation}
En isolant $v_{pb}$, on obtien un valeur de :
\begin{equation}
v_{pb} = 5,59m/s
\end{equation}
À l'aide de cette vitesse, on doit règler la minutrie en sorte à ce que le participant ait quitté la plateform au complet avant que celle-ci s'ouvre.
\begin{equation}
\delta t_m = \frac{l_{trappe}}{v_{pb}}
\end{equation}
\begin{equation}
\delta t_m = \frac{3m}{5,59m/s} \approx{0,54}
\end{equation}
Et selon les standard imposés, la minutrie devait avoir une marge de manoeuvre de $0,02s$.
\begin{equation}
\delta t_m \approx{0,54 - 0,02} = 0,52sec
\end{equation}
\subsection{Ballon non attrapé}
Dans cette situation, le participant entre en collision avec le ballon sans l'attrapé. La collision entre le ballon-mousse et le participant à ce moment là est une collision plastique. Selon les requis du devis de WOQ, nous considérons le coefficient de récupération de $0,8$.
\begin{equation}
e <= 0,8 = \frac{V'_{b_n} - V'_{p_n}}{V_{p_n} - V_{b_n}}
\end{equation}
suite a des manipulations algébrique, le résultat est:
\begin{equation}
{V_{p_n} - V_{b_n}} = 
\end{equation}

\section{Design de la minuterie}
\subsection{Calculs}

\section{Design du Coussin-Tampoline}
\subsection{Calculs}

\section{Design du Bassin}
\subsection{Calculs}

\section{Conclusion}


\end{document}